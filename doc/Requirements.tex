\documentclass[12pt]{article}

% Author's up-front packages
\usepackage[T1]{fontenc}
\usepackage[utf8]{inputenc}

% Packages on template
\usepackage{amsmath, mathtools}
\newcommand\norm[1]{\lVert#1\rVert}
\newcommand\normx[1]{\Vert#1\Vert}
\usepackage{amsfonts}
\usepackage{amssymb}
\usepackage{graphicx}
\usepackage{colortbl}
\usepackage{xr-hyper}
\usepackage{hyperref}
\usepackage{longtable}
\usepackage{xfrac}
\usepackage{tabularx}
\usepackage{float}
\usepackage{siunitx}
\usepackage{booktabs}
\usepackage{caption}
\usepackage{pdflscape}
\usepackage{afterpage}

% ---- Author's choice to remove them ----
%\usepackage[round]{natbib}
%\usepackage{refcheck}

% Author's packages
\usepackage{cite}
\usepackage{indentfirst}
\usepackage{cleveref}
\usepackage{float}
\usepackage{soul}
\newcommand\Definition[1]{{\bfseries \ul{#1}}}
\usepackage{xcolor}
\definecolor{shadecolorIM}{RGB}{0,200,0}
\definecolor{shadecolorT}{RGB}{0,191,255}
\definecolor{shadecolorDD}{RGB}{186,85,211}
\definecolor{shadecolorR}{RGB}{255,0,0}

\hypersetup{
    %bookmarks=true,			% show bookmarks bar?
    colorlinks=true,			% false: boxed links; true: colored links
    linkcolor=red,				% color of internal links (change box color with 
%linkbordercolor)
    citecolor=green,        % color of links to bibliography
    filecolor=magenta,      % color of file links
    urlcolor=blue           % color of external links
}

% For easy change of table widths
\newcommand{\colZwidth}{1.0\textwidth}
\newcommand{\colAwidth}{0.13\textwidth}
\newcommand{\colBwidth}{0.82\textwidth}
\newcommand{\colCwidth}{0.1\textwidth}
\newcommand{\colDwidth}{0.05\textwidth}
\newcommand{\colEwidth}{0.8\textwidth}
\newcommand{\colFwidth}{0.17\textwidth}
\newcommand{\colGwidth}{0.5\textwidth}
\newcommand{\colHwidth}{0.28\textwidth}


% Used so that cross-references have a meaningful prefix
\newcommand{\progname}{STEMMoireRec}

\usepackage{fullpage}

%Set the custom referencing system (author's initiative)
	% Goal Statement
\newtheorem{GS}{GS}
\crefname{GS}{GS}{GSs}
	% Assumption
\newtheorem{A}{A}
\crefname{A}{A}{As}
	% Theoretical Model
\newtheorem{T}{T}
\crefname{T}{T}{Ts}
	% Data Definition
\newtheorem{DD}{DD}
\crefname{DD}{DD}{DDs}
	% Data Constraints
\newtheorem{DC}{DC}
\crefname{DC}{DC}{DCs}
	% Instance Model
\newtheorem{IM}{IM}
\crefname{IM}{IM}{IMs}
	% General Definition
\newtheorem{GD}{GD}
\crefname{GD}{GD}{GDs}
	% Requirement
\newtheorem{R}{R}
\crefname{R}{R}{Rs}
	% Non Functional Requirement
\newtheorem{NR}{NR}
\crefname{NR}{NR}{NRs}
	% Likely change
\newtheorem{LC}{LC}
\crefname{LC}{LC}{LCs}

\begin{document}


\title{Software Requirements Specifications (SRS) \\
STEM Moir{\'e} GPA} 
\author{Alexandre Pofelski \\
		macid: pofelska \\
		github: slimpotatoes}
\date{\today}

\maketitle

\clearpage

\pagenumbering{roman}
\tableofcontents

\clearpage

\section{Revision History}

\begin{table}[h]
\caption{\bf Revision History}
\begin{tabularx}{\textwidth}{p{3cm}p{2cm}X}
\toprule {\bf Date} & {\bf Version} & {\bf Notes}\\
\midrule
xx/xx/xxxx & 1.0 & First Draft\\

\bottomrule
\end{tabularx}
\end{table}

\clearpage

\section{Reference Material}

\subsection{Table of Units}

Throughout this document SI 
(\href{<https://physics.nist.gov/cuu/Units/index.html>}{Syst\`{e}me 
Internationale d'Unit\'{e}s}) is employed as the unit system. In addition to the 
basic units, several derived units are used as described below.  For each unit, 
the symbol is given followed by a description of the unit and the SI name.\par 
\bigskip

\renewcommand{\arraystretch}{1.2}
%\begin{table}[ht]
  \noindent \begin{tabular}{l l l} 
    \toprule		
    \textbf{Symbol} & \textbf{Base quantity} & \textbf{Name SI}\\
    \midrule 
    \si{\metre} & length & metre\\
    \si{\per\metre} & reciprocal meter & wave number\\
    \bottomrule
  \end{tabular}
  %	\caption{Provide a caption}
%\end{table}

\subsection{Table of Symbols}

The table that follows summarizes the symbols used in this document along with
their units if applicable.

%\renewcommand{\arraystretch}{1.2}
\noindent 
\begin{longtable*}{l l p{12cm}} 
\toprule
\textbf{Symbol} & \textbf{Unit} & \textbf{Description}\\
\midrule
\endhead
$\mathcal{B}_2$ & & 2D orthonormal base  in real space \\
$\mathcal{B}_{2^{*}}$ & & 2D orthonormal base in reciprocal space \\
$\mathcal{B}_3$ & & 3D orthonormal base in real space \\
$\mathcal{B}_{3^{*}}$ & & 3D orthonormal base in reciprocal space \\
$\mathcal{B}_C$ && 3D crystal lattice base \\
$\mathcal{B}_{C^{*}}$ && 3D reciprocal crystal lattice base \\
$\mathcal{B}_S$ & & 3D orthonormal base in real space defined by the 2D orthonormal sampling scheme\\
 $\mathcal{B}_{S}^{*}$  && 3D orthonormal base in reciprocal space defined by the 2D orthonormal sampling scheme \\
$\delta$ & & Dirac delta function \\
$\mathcal{FT}$ & & Fourier transform \\
$\Gamma_{p^{X}}$ & & Spatial frequency range $[-1/2p, 1/2p]^{X}$\\
$\overrightarrow{g_{hkl}}$ & \si{\per\nano\meter} & wave vector associated with $(hkl)$ Miller indices \\
$\overrightarrow{g_{hkl}}^{C}$ & \si{\per\nano\meter} & crystalline wave 
vector with $(hkl)$ Miller indices \\
$\overrightarrow{g_{hkl}}^{M}$ & \si{\per\nano\meter} & Moir{\'e} wave vector of associated with $(hkl)$ Miller indices\\
$G^{S}_{R}$ && List of resolved and allowed reflections \\
$i$ & & Imaginary unit \\
$I$ & & Intensity (or number of counts) \\
$\mathbb{I}$ & & Sub-set of $\mathbb{R}$ representing the position of the pixels 
in an image\\
$I_{C_{\text{ref}}}$ & & 2D array representing the reference crystal structure 
\\
$I_{\mathit{SMH}_{\text{exp}}}$ & & 2D array representing the experimental SMH\\
$I_{\mathit{SMH}_{\text{sim}}}$ & & 2D array representing the simulated SMH\\
$M_j$ & & Mask function in Fourier space including the $j^{\text{th}}$ wave 
vector\\
$\mathbb{N}$ & & Set of natural numbers\\
$\vec{\nu}$ & \si{\per\nano\meter} & Vector position in Fourier space\\
$O$ & & Origin of the coordinate system \\
$p$ & \si{\nano\meter} & Pixel size of the experimental SMH 
$I_{\mathit{SMH}_{\text{exp}}}$ \\
$\overrightarrow{q_{n_j,m_j}}$ & & Sampling vector $\vec{q} \in Q$ of couple 
$(n_j,m_j)$\\
$Q$ & & Set such that $Q=\{\forall (n,m) \in \mathbb{Z}^{2}, 
\vec{q}=n\vec{u_x}+m\vec{u_y}\}$ \\
$\vec{r}$ & \si{\nano\meter} & Vector position \\
$R$ & \si{\nano\meter} & Resolution of the STEM probe \\
$\mathbb{R}$ & & Set of real numbers \\
$\sigma$ &&  Lattice centering \\
$\mathit{SMH}_{\text{exp}}$ & & 2D array representing the experimental STEM 
Moir{\'e} Hologram \\
$\mathit{SMH}_{\text{sim}}$ & & 2D array representing the simulated STEM 
Moir{\'e} Hologram \\
$\wedge$ & & Mathematical symbol for "and"\\
$\mathbb{Z}$ & & Set of integer numbers\\
\bottomrule
\label{table_symbols_SRS}
\end{longtable*}

\subsection{Abbreviations and Acronyms}

\renewcommand{\arraystretch}{1.2}
\begin{tabular}{l l} 
  \toprule		
  \textbf{symbol} & \textbf{description}\\
  \midrule 
  A & Assumption\\
  AU & Arbitrary Unit\\
  DC & Data Constraint \\
  DD & Data Definition\\
  EM & Electron Micrograph \\
  GD & General Definition\\
  GS & Goal Statement\\
  IM & Instance Model\\
  LC & Likely Change\\
  PS & Physical System Description\\
  NR & Non functional Requirement\\
  R & Requirement\\
  SMH & STEM Moir{\'e} Hologram \\
  SRS & Software Requirements Specification\\
  STEM & Scanning Transmission Electron Microscopy \\
  T & Theoretical Model\\
  \bottomrule
  \label{table_acro_SRS}
\end{tabular}

\newpage
\pagenumbering{arabic}

\section{Specific System Description}
\subsection{Problem Description} \label{Sec_pd}

BLABLA. Terminologies and the physical system are 
described below.

\subsubsection{Terminology and Definitions}

\begin{itemize}

\item \textbf{3D Cartesian coordinate system}: orthonormal coordinate system 
model by the base $\mathcal{B}=(O,\vec{u_x},\vec{u_y},\vec{u_z} )$ with $O$ representing 
the origin and $M$ a point with coordinate the $(x,y,z)$, such that any vector 
$\vec{r}=\overrightarrow{\mathit{OM}}$ can be expressed as the following :
\begin{equation}
\forall (x,y,z) \in \mathbb{R}^{3}, \vec{r}=x\vec{u_x}+y\vec{u_y}+z\vec{u_z}
\end{equation}
\item \textbf{Pixel}: xxx
\item \textbf{Electron Micrograph (EM)}: 2D array collected in an electron 
microscope representing the number of electron crossing the sample (intensity) 
at each pixel location.
\item \textbf{Scanning grid}: set representing the succession of the STEM probe 
positions when collecting the STEM EM. Equivalently the scanning grid represents 
the relative position of the pixel with respect to the sample when acquiring the 
EM. A simplified version of the 
STEM EM formation can be visualized in \cref{fig:STEM_imaging_Fig}. The 
positions of the STEM probe are located at the intersection of the black grid 
lines.
%\begin{figure}[H]
%\begin{center}
%\includegraphics[scale=0.7]{Figures/STEM_imaging_Fig.png}
%\caption{(Left) Schematic of the STEM EM formation with the STEM probe scanning 
%the sample at each intersection of the grid lines. The electrons crossing the 
%sample are collected on the detector and counted during the acquisition time. 
%(Right) STEM EM on a pure silicon sample revealing its atomic structure. In the 
%inset is highlighted the type of data the STEM EM corresponds to which is a 2D 
%array with the intensity collected on each pixel.}
%\label{fig:STEM_imaging_Fig}
%\end{center}
%\end{figure}
\item \textbf{Crystal lattice}: Periodic arrangement of atoms forming matter.
\item \textbf{STEM Moir{\'e} hologram (SMH)}: EM collected in STEM and resulting 
from the interference between the scanning grid and the crystal lattice. 
\end{itemize}

\subsubsection{Physical System Description}

The physical system of \progname{}, as shown in \cref{fig:system_description}, 
includes the following elements:
\begin{itemize}
\item The STEM Moir{\'e} hologram as the results of the interaction between the 
scanning grid and the crystal periodicity of the sample.
\item Physical inputs provided by the user to convert a STEM Moir{\'e} hologram 
into strain and rotation maps.
\end{itemize}

%\begin{figure}[H]
%\begin{center}
%\includegraphics[scale=0.65]{Figures/System_description.png}
%\caption{Physical system of \progname{} highlighted in the yellow area.}
%\label{fig:system_description}
%\end{center}
%\end{figure}

\subsubsection{Goal Statements}
\noindent Given the system description, the goal statement is:
\begin{GS}
\normalfont Reconstruct an oversampled image from a STEM Moir{\'e} hologram
\label{GS_1}
\end{GS}

\subsection{Solution Characteristics Specification}

\subsubsection{Assumptions}

\begin{A}
\normalfont The resolution of the microscope cannot resolve any spatial frequency higher than 
$g_{j_{\text{lim}}}$. 
\label{A_1}
\end{A}
\begin{A}
\normalfont Uniform orthogonal and uniform samplers are considered.
\label{A_2}
\end{A}
\begin{A}
\normalfont Blabla
\label{A_3}
\end{A}
\begin{A}
\label{A_4}
\normalfont Blabla
\end{A}


\subsubsection{Theoretical Models}\label{sec_theoretical}

\begin{T}
\label{T_1}
\noindent\colorbox{shadecolorT}{\normalfont \textbf{2D periodic sampling}}
\normalfont
\begin{itemize}
\item \underline{Equation}: \Cref{eq:sampling_simplified}
\item \underline{Description}: In the 2D Cartesian coordinate system 
$\mathcal{B}_2$, the scanning grid can be seen as sampler $S$ sampling a 
continuous function $f$. In the context of the \progname{} project, the sampler 
is set to be periodic with the same periodicity $p$ (called pixel size) in both 
$x$ and $y$ directions (2D Dirac comb). The resulting sampled version $f_S$ of 
$f$ can be represented as the following with $\delta$ representing the Dirac 
function:
\begin{equation}
\begin{gathered}
\forall (x,y) \in \mathbb{R}^{2}, f_S(x,y)=S(x,y)\times f(x,y) \\
\forall (x,y) \in \mathbb{R}^{2}, 
f_S(x,y)=\sum_{n=-\infty}^{n=+\infty}\sum_{m=-\infty}^{m=+\infty}\delta(x-np,y-mp)\times 
f(x,y)
\end{gathered}
\label{eq:sampling}
\end{equation}
For shorter notations, it is possible to define a set Q as follows $Q=\{\forall 
(n,m) \in \mathbb{Z}^{2}, \vec{q}=n\vec{u_x}+m\vec{u_y}\}$ and thus simplify 
\cref{eq:sampling}
\begin{equation}
\forall (x,y) \in \mathbb{R}^{2}, f_S(\vec{r})=\sum_{q\in 
Q}\delta(\vec{r}-p\vec{q}) f(\vec{r})
\label{eq:sampling_simplified}
\end{equation}
\item \underline{Source}: \cite{Gonzalez_1992}
\item \underline{Ref by}: \cref{DD_2}, \cref{IM_1}, \cref{IM_4}
\end{itemize}
\end{T}

\begin{T}
\label{T_2}
\noindent\colorbox{shadecolorT}{\normalfont \textbf{Crystal lattice}}
\normalfont
\begin{itemize}
\item \underline{Equation}: \Cref{eq:crystal_1}
\item \underline{Description}: In the 3D crystal lattice coordinate system $\mathcal{B}_C$, the periodic arrangement of a crystalline material in reciprocal space is as follows with $\vec{a}$, $\vec{b}$ and $\vec{c}$ describing the crystal unit cell:
\begin{equation}
\forall (u,v,w) \in \mathbb{Z}^{3}, \ \overrightarrow{OM}(u,v,w) = \vec{r_C} = u \vec{a} + v \vec{b} + w \vec{c} 
\end{equation}
\item \underline{Source}: xxx
\item \underline{Ref by}: \cref{DD_3}, \cref{IM_3}, \cref{IM_5}
\end{itemize}
\end{T}

\begin{T}
\label{T_3}
\noindent\colorbox{shadecolorT}{\normalfont \textbf{Reciprocal crystal lattice}}
\normalfont
\begin{itemize}
\item \underline{Equation}: \Cref{eq:reciprocal_crystal_1}
\item \underline{Description}: Reciprocal crystal lattice
\begin{equation}
\begin{cases}
\vec{a}^{*} = \frac{\vec{b}\times \vec{c}}{\vec{a}\cdot(\vec{b}\times \vec{c})} \\
\vec{b}^{*} = \frac{\vec{c}\times \vec{a}}{\vec{b}\cdot(\vec{c}\times \vec{a})} \\
\vec{c}^{*} = \frac{\vec{a}\times \vec{b}}{\vec{c}\cdot(\vec{a}\times \vec{b})}
\end{cases}
\end{equation}
leading the to following reciprocal lattice base $\mathcal{B}_{C^{*}}=(\vec{a}^{*}, \vec{b}^{*}, \vec{c}^{*})$ and the expression of the reciprocal lattice $\vec{OM}(h,k,l) = h\vec{a}^{*} + k\vec{b}^{*} + l\vec{c}^{*}$
\item \underline{Source}: xxx
\item \underline{Ref by}: \cref{DD_3}, \cref{IM_3}, \cref{IM_5}
\end{itemize}
\end{T}

\begin{T}
\label{T_4}
\noindent\colorbox{shadecolorT}{\normalfont \textbf{Reconstruction of a sparse bandwidth-limited periodic function}}
\normalfont
\begin{itemize}
\item \underline{Equation}: \Cref{eq:strain_2}
\item \underline{Description}: Bla
\item \underline{Source}: Our paper
\item \underline{Ref by}: \cref{DD_3}, \cref{IM_3}, \cref{IM_5}
\end{itemize}
\end{T}

\subsubsection{Data Definitions}\label{sec_datadef}

\renewcommand{\labelitemi}{$\star$}

\begin{DD}
\label{DD_1}
\noindent\colorbox{shadecolorDD}{\normalfont \textbf{Coordinate of the atoms $ \overrightarrow{OM}(x,y,z)$ in the $\mathcal{B}_3$ base, \\ aligning $\vec{u_z}$ to $\vec{c}$ and using $\vec{u_y}$ in the $(\vec{b}, \vec{c})$ plane}}
\normalfont
\begin{itemize}
\item \underline{Equation}:  $ \overrightarrow{OM} = 
\begin{bmatrix}
 x \\
y \\
z 
\end{bmatrix} =  \mathcal{M_{\mathcal{B}_C\rightarrow \mathcal{B}_3}} 
\begin{bmatrix}
u \\
v \\
w 
\end{bmatrix} \\ 
\text{with} \ \mathcal{M_{\mathcal{B}_C\rightarrow \mathcal{B}_3}}  = 
\begin{bmatrix}
a\sin{\beta}\sin{\gamma^{*}} & 0 & 0 \\
a\sin{\beta}\cos{\gamma^{*}} & b\sin{\beta} & 0 \\
a\cos{\beta} & b\cos{\beta} & c
\end{bmatrix} \ \text{and} \ 
\begin{cases}
\cos{\gamma^{*}} =  \frac{\cos{\alpha}\cos{\beta} - cos{\gamma}}{\sin{\alpha}\sin{\beta}} \\
\sin{\gamma^{*}} = 1 - (\cos{\gamma^{*}})^{2}
\end{cases}$
\item \underline{Description}: 
\item \underline{Source}: Regarder dans le folder
\item \underline{Ref by}: 
\end{itemize}
\end{DD}

\begin{DD}
\label{DD_2}
\noindent\colorbox{shadecolorDD}{\normalfont \textbf{Coordinate of the reflection $ \overrightarrow{OM}(\lambda,\mu,\nu)$ in the $\mathcal{B}_3$ base, \\ aligning $\vec{u_x}$ to $\vec{a}^{*}$ and using $\vec{u_y}$ in the $(\vec{a}^{*}, \vec{b}^{*})$ plane}}
\normalfont
\begin{itemize}
\item \underline{Equation}:  $ \overrightarrow{OM} = 
\begin{bmatrix}
\lambda \\
\mu \\
\nu 
\end{bmatrix} =  \mathcal{M_{\mathcal{B}_{C^{*}}\rightarrow \mathcal{B}_3}} 
\begin{bmatrix}
h \\
k \\
l 
\end{bmatrix} \\ 
\text{with} \ \mathcal{M_{\mathcal{B}_{C^{*}}\rightarrow \mathcal{B}_3}}  = 
(\mathcal{M_{\mathcal{B}_{C}\rightarrow \mathcal{B}_3}}^{T})^{-1} $
\item \underline{Description}: 
\item \underline{Source}: Regarder dans le folder
\item \underline{Ref by}: 
\end{itemize}
\end{DD}

\begin{DD}
\label{DD_3}
\noindent\colorbox{shadecolorDD}{\normalfont \textbf{Type of crystal}}
\normalfont
\begin{itemize}
\item \underline{Equation}:  $
\begin{cases}
(a=b=c) \wedge (\alpha=\beta=\gamma = \ang{90}) \rightarrow \text{cubic}\\
(a=b=c) \wedge (\alpha=\beta=\ang{90} \ \gamma = \ang{120}) \rightarrow \text{hexagonal} \\
(a=b\neq c) \wedge (\alpha=\beta=\gamma = \ang{90}) \rightarrow \text{tetragonal}\\
(a\neq b\neq c) \wedge (\alpha=\beta=\gamma = \ang{90}) \rightarrow \text{orthorombic}\\
(a\neq b\neq c) \wedge (\alpha=\beta=\ang{90} \ \gamma \neq \ang{90}) \rightarrow \text{monoclinic}\\
(a\neq b\neq c) \wedge (\alpha\neq\beta\neq\gamma)  \rightarrow \text{triclinic}
\end{cases}$ \\
Permutations of $(a,b,c)$ and $(\alpha, \beta, \gamma)$ are also considered 
\item \underline{Description}: 
\item \underline{Source}: Regarder dans le folder
\item \underline{Ref by}: 
\end{itemize}
\end{DD}

\begin{DD}
\label{DD_4}
\noindent\colorbox{shadecolorDD}{\normalfont \textbf{Simplified reflection $(h,k,l)$ selection rules $\Omega$ }}
\normalfont
\begin{itemize}
\item \underline{Equation}:  $
\begin{cases}
\text{Primitive} \rightarrow (h,k,l) \in \mathbb{Z}^{3} \\
\text{Body-centered} \rightarrow h+k+l=2n+1 \ \text{with} \ n \in \mathbb{Z} \\
\text{Face-centered} \rightarrow h,k,l \  \text{all odd or all even} \\
\text{Face-centered diamond} \rightarrow h,k,l \ \text{all odd or} \  h,k,l \ \text{all even and} \ h+k+l=4n \ \text{with} \ n \in \mathbb{Z} \\
\text{Hexagonal closed packed} \rightarrow l  \ \text{even  or} \  h+2k\neq 3n \ \text{with} n \in \mathbb{Z}\\
\end{cases}$
\item \underline{Description}: Simplified selection rules for existence of the $(h,k,l)$ reflection for some crystal structure $S$. The only way to be generic is to consider the lattice centering (P, I, A, B, C, F) with the crystal structure to generate the selection rule or look at the group space directly.
\item \underline{Source}: Regarder dans le folder
\item \underline{Ref by}: 
\end{itemize}
\end{DD}

\subsubsection{Instance Models} \label{sec_instance}    

\renewcommand{\labelitemi}{$-$}

\begin{IM}
\label{IM_1}
\noindent\colorbox{shadecolorIM}{\normalfont \textbf{List the allowed crystal reflections resolved by the STEM probe}}
\normalfont
\begin{itemize}
\item \textbf{Input}: $a, b, c, \alpha, \beta, \gamma, S, R$
\item \textbf{Output}:$G^{\sigma}_{R}$
\item \textbf{Description} : For all the resolved reflection ($\norm{\overrightarrow{g_{hkl}}}_{\mathcal{B}_{3}^{*}} < R$), regroup all the allowed reflections ($S \in \Omega$) in $G^{\sigma}_{R}$ expressed in the base $\mathcal{B}_{3}^{*}$.
\item \textbf{Source}: xxx
\item \textbf{Ref by}: xxxx
\end{itemize}
\end{IM}

\begin{IM}
\label{IM_2}
\noindent\colorbox{shadecolorIM}{\normalfont \textbf{Project the crystal reflections in the 3D sampling base $\mathcal{B}_{{S}^{*}}$}}
\normalfont
\begin{itemize}
\item \textbf{Input}: $G^{\sigma}_{R}$,  $\mathcal{B}_{S}=(\vec{s_x}, \vec{s_y}, \vec{s_z})$
\item \textbf{Output}:$G^{\sigma}_{R}$
\item \textbf{Description} : For all the reflections in $G^{\sigma}_{R}$, apply the base transformation from $\mathcal{B}_{{3}^{*}}$ to $\mathcal{B}_{{S}^{*}}$ using the transformation matrix $\mathcal{M}_{\mathcal{B}_{{3}^{*}}\rightarrow \mathcal{B}_{{S}^{*}}}$. 
\begin{equation}
g_{hkl} = \begin{bmatrix}
\lambda_s \\
\mu_s \\
\nu_s 
\end{bmatrix} =  \mathcal{M}_{\mathcal{B}_{{3}^{*}}\rightarrow \mathcal{B}_{{S}^{*}}} 
\begin{bmatrix}
\lambda \\
\mu \\
\nu 
\end{bmatrix}
\end{equation}
\item \textbf{Source}: xxx
\item \textbf{Ref by}: xxxx
\end{itemize}
\end{IM}

\begin{IM}
\label{IM_3}
\noindent\colorbox{shadecolorIM}{\normalfont \textbf{Simulate the STEM Moir{\'e} hologram imaging formation}}
\normalfont
\begin{itemize}
\item \textbf{Input}: xxxxx
\item \textbf{Output}:xxxx

\item \textbf{Source}: xxx
\item \textbf{Ref by}: xxxx
\end{itemize}
\end{IM}

\begin{IM}
\label{IM_4}
\noindent\colorbox{shadecolorIM}{\normalfont \textbf{Mask and transform the Moire reflection into the crystalline one}}
\normalfont
\begin{itemize}
\item \textbf{Input}: xxxxx
\item \textbf{Output}:xxxx

\item \textbf{Source}: xxx
\item \textbf{Ref by}: xxxx
\end{itemize}
\end{IM}

\begin{IM}
\label{IM_5}
\noindent\colorbox{shadecolorIM}{\normalfont \textbf{Reconstruct the oversample electron micrograph}}
\normalfont
\begin{itemize}
\item \textbf{Input}: xxxxx
\item \textbf{Output}:xxxx

\item \textbf{Source}: xxx
\item \textbf{Ref by}: xxxx
\end{itemize}
\end{IM}

\end{document}